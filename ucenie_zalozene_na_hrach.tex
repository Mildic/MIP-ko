% Metódy inžinierskej práce

\documentclass[10pt,twoside,slovak,a4paper]{article}

\usepackage[slovak]{babel}
%\usepackage[T1]{fontenc}
\usepackage[IL2]{fontenc} % lepšia sadzba písmena Ľ než v T1
\usepackage[utf8]{inputenc}
\usepackage{graphicx}
\usepackage{url} % príkaz \url na formátovanie URL
\usepackage{hyperref} % odkazy v texte budú aktívne (pri niektorých triedach dokumentov spôsobuje posun textu)

\usepackage{cite}
%\usepackage{times}

\pagestyle{headings}

\title{Učenie založené na hrách\thanks{Semestrálny projekt v predmete Metódy inžinierskej práce, ak. rok 2022/23, vedenie: Zuzana Špitálová}} % meno a priezvisko vyučujúceho na cvičeniach

\author{Miloš Krupka\\[2pt]
	{\small Slovenská technická univerzita v Bratislave}\\
	{\small Fakulta informatiky a informačných technológií}\\
	{\small \texttt{xkrupka@stuba.sk}}
	}

\date{\small 6. november 2022} % upravte



\begin{document}

\maketitle

\begin{abstract}
Cieľom tohto článku je poukázať na vplyv hier a herných prvkov v procese nadobúdania nových poznatkov, zručností a dosahovania cieľov. Budeme pozorovať spôsoby učenia sa u detí v predškolskom veku, žiakov, a pracujúcich ľudí.  Tieto kategórie boli zvolené zámerne, kvôli tomu, že sa jedná o vývin ľudskej bytosti, pričom tá v každej fáze využíva určité prostriedky, vďaka ktorým sa rozširujú jej obzory.  V závere bude jasne zhrnuté, aký vplyv má učenie založené na hrách a herných prvkoch. Či sú zručnosti nadobudnuté aj vďaka hrám využiteľné v 21. storočí a teda majú naozaj opodstatnenie alebo sú naopak kontraproduktívne a prispievú k nižšej motivácii jednotlivca, či skupiny ľudí.
\ldots
\end{abstract}



\section{Úvod}

Hra je neodlúčiteľnou súčasťou ľudského života. Stretávame sa s ňou už od útleho detstva a sprevádza nás každou etapou nášho života. Je dokonca staršia ako ľudská história (sekcia 1). Pri uvažovaní nad slovom hra nám napadne hneď niekoľko vecí, ktoré sa spájajú s týmto slovom. Hra ako forma zábavy, prípadne konkrétne príklady ako hra na hudobný nástroj, schovávačka, spoločenská hra, ale aj tá virtuálna, či hazardná. Hra sa nám spája so zábavou, ale svoje miesto nachádza aj vo vzdelávacom procese (sekcia 2). Jednotlivé časti pojednávajú o učení za pomoci hier a herných prvkov v etapách života (sekcia 2). Ovplyvňujú vzdelávanie na školách, profesijný, ale aj súkromný život.  Záverečné zhrnutie nájdeme v časti (záver).



\section{Nejaká časť} \label{nejaka}

(história hry):
Prvé formy hry môžeme pozorovať v zvieracej ríši. Mláďatá zvierat sa učia žiť vo zvieracom spoločenstve(svorka, stádo, kŕdeľ), zaobstarávať si potravu a budovať priebojnosť. Prežije len ten najsilnejší. Mláďatá leva sa učia žiť vo svorke, loviť si potravu a tým nadobúdajú nové schopnosti. Môžeme u nich pozorovať aj formy hry, kedy sa mláďatá naťahujú o kus potravy alebo sa naháňajú a bijú labkami. Sú to ich prvé kroky k osvojeniu zručností, ktoré neskôr využijú pri love a žitiu vo svorke. Podobné obrazy nám ponúka aj história človeka. Môžeme v nej pozorovať momenty, pri ktorých hra a jej aspekty ovplyvnili skúsenosti jedinca a jeho pohľad na svet. Hra však sama o sebe je formou zábavy a ak je dobre využitá, stáva sa z nej cesta k poznaniu nových vecí, teda k učeniu a následnému porozumeniu. 

\subsection{Ako sa učia malé deti} \label{nejaka:nieco}
	Detstvo býva obdobím neustáleho spoznávania sveta, objavovania vlastných schopností a záľub. Deti vďaka pravidelnému kontaktu s rôznymi hračkami (hrami) spoznávajú farby, číslice, zvieratá, učia sa hovoriť a socializujú sa. Takto si budujú vlastný obraz sveta a nadobúdajú skúsenosti. Takže sa jedná o učenie, ktoré je založené na hrách. Hry sa v takomto učení stávajú nevtieravými  pomocníkmi. Na potenciál využitia hier v procese nadobúdania nových schopností a zručností poukazoval v našej histórii Ján Amos Komenský, ktorý nebol spokojný s úrovňou vzdelávania. Chcel žiakov zapojiť do dramatických hier na rôzne historické témy, aby sa v nich prebudila túžba po učení. Zakladal si na tom, že hra je v procese učenia veľmi, najmä kvôli tomu, že motivuje jednotlivca k opusteniu doterajšej pasivity. Podľa Komenského je hra rovnako dôležitá ako spánok, či iné ľudské potreby. Hra je pre človeka prirodzená. Pri deťoch to platí o to viac, pretože sa s hrou stretávajú doma, v škôlke, ale aj v škole. Je pre nich základným prostriedkom pre vnímanie sveta a život v spoločenstve iných detí. Deti sa učia skúsenosťou a tú majú z hrania sa. Následne sa pokúšajú získanú skúsenosť aplikovať na nové problémy. Zoberme si príklad s legom. Deti sa hrajú s legom a stavajú si z neho rôzne veci. Zistia, že vysoká stavba spadne, ak ju prevážia a následne hľadajú spôsob, ako ju spevniť. Neskôr sa pokúsia danú skúsenosť využiť pri výbere pohára zo skrine, berúc ohľad na to, že skriňa je vysoko. Ak sa chcú dostať k poháru, potrebujú využiť stoličku, ktorú si v prípade nestability zastabilizujú. Keď sa na ňu postavia a vyberú si pohár, vyriešili svoj problém.  Deti sa v škôlke učia za pomoci hier a hravých prostriedkov, ktoré im prinášajú zážitkovú formu učenia. Hrajú sa s inými deťmi. Učia sa básničky, súťažia v športových aktivitách, tancujú,  spievajú si. Vďaka tomu spoznávajú nové veci a berú to ako zábavu, čo prispieva k radosti zo skúseností. 

\subsection{Ako sa učia deti v škole} \label{nejaka:nieco}
Na prvom stupni sa deti stavajú do pozície, v ktorej si postupne začnú uvedomovať, že prechádzajú od hry k učeniu. Učia sa čítať, písať a počítať, aj keď niektoré základné veci sa učili už v predprimárnom vzdelávaní. Pedagógovia využívajú interaktívnu tabuľu, tablety, počítače. Tie im slúžia ako pomôcky, pri ktorých môžu aktívne pracovať s každým dieťaťom. Majú k dispozícii rôzne aplikácie, ktoré skvalitňujú proces výučby. Sami si spomíname na hry, na ktorých sme si precvičovali slovíčka z anglického jazyka alebo sme rátali príklady, aby sme nahrali, čo najlepšie skóre.** Avšak s príchodom minecraftu vznikla iniciatíva, ktorá viedla k vytvoreniu vzdelávacej verzie Minecraft: Education Edition.**** Dnes túto verziu využívajú niektorí pedagógovia pri výučbe informatiky. Samotné hranie hry viedlo k pozitívnej spätnej väzbe zo strany vyučujúcich aj detí. Využitie minecraftu prispelo k budovaniu cenných zručností ako kritické myslenie, riešenie problémov, nadhľad, tímovosť, čo sú tzv. mäkké zručnosti (soft skills). Deti pri hre čelia problémom, ktoré riešia rôznymi spôsobmi. 

Z obr.~\ref{f:rozhod} je všetko jasné. 

\begin{figure*}[tbh]
\centering
%\includegraphics[scale=1.0]{diagram.pdf}
Aj text môže byť prezentovaný ako obrázok. Stane sa z neho označný plávajúci objekt. Po vytvorení diagramu zrušte znak \texttt{\%} pred príkazom \verb|\includegraphics| označte tento riadok ako komentár (tiež pomocou znaku \texttt{\%}).
\caption{Rozhodujúci argument.}
\label{f:rozhod}
\end{figure*}



\section{Iná časť} \label{ina}

Základným problémom je teda\ldots{} Najprv sa pozrieme na nejaké vysvetlenie (časť~\ref{ina:nejake}), a potom na ešte nejaké (časť~\ref{ina:nejake}).\footnote{Niekedy môžete potrebovať aj poznámku pod čiarou.}

Môže sa zdať, že problém vlastne nejestvuje\cite{Coplien:MPD}, ale bolo dokázané, že to tak nie je~\cite{Czarnecki:Staged, Czarnecki:Progress}. Napriek tomu, aj dnes na webe narazíme  na na všelijaké pochybné názory\cite{PLP-Framework}. Dôležité veci možno \emph{zdôrazniť kurzívou}.


\subsection{Nejaké vysvetlenie} \label{ina:nejake}

Niekedy treba uviesť zoznam:

\begin{itemize}
\item jedna vec
\item druhá vec
	\begin{itemize}
	\item x
	\item y
	\end{itemize}
\end{itemize}

Ten istý zoznam, len číslovaný:

\begin{enumerate}
\item jedna vec
\item druhá vec
	\begin{enumerate}
	\item x
	\item y
	\end{enumerate}
\end{enumerate}


\subsection{Ešte nejaké vysvetlenie} \label{ina:este}

\paragraph{Veľmi dôležitá poznámka.}
Niekedy je potrebné nadpisom označiť odsek. Text pokračuje hneď za nadpisom.



\section{Dôležitá časť} \label{dolezita}




\section{Ešte dôležitejšia časť} \label{dolezitejsia}




\section{Záver} \label{zaver} % prípadne iný variant názvu



%\acknowledgement{Ak niekomu chcete poďakovať\ldots}


% týmto sa generuje zoznam literatúry z obsahu súboru literatura.bib podľa toho, na čo sa v článku odkazujete
\bibliography{literatura}
\bibliographystyle{plain} % prípadne alpha, abbrv alebo hociktorý iný
\end{document}